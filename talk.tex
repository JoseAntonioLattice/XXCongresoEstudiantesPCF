\documentclass[11pt]{beamer}
\usetheme{Bergen}
\usepackage[utf8]{inputenc}
\usepackage{amsmath}
\usepackage{amsfonts}
\usepackage{amssymb}
\usepackage{graphicx}
\usepackage{hyperref}
\usepackage{fontawesome}
\usepackage{dsfont}
\author{José Antonio García Hernández}
\title{Métodos Monte Carlo en Teorías Cuánticas de Campos (Lattice QFT)}
\setbeamercovered{transparent} 
\setbeamertemplate{navigation symbols}{} 
\logo{} 
\institute{Instituto de Ciencias Nucleares, UNAM} 
\date{29 noviembre 2024} 
\subject{Lattice Field Theories} 
\begin{document}

\begin{frame}
\titlepage
\end{frame}

%\begin{frame}
%\tableofcontents
%\end{frame}

\begin{frame}{Formalismo de la integral funcional}

    Suma sobre configuraciones del campo $\phi$. Integral funcional
    $$Z = \int \mathcal{D}\phi \, e^{iS[\phi]},$$
    donde $S[\phi]$ es la acción de la configuración
    $$ S[\phi] = \int d^4 x \, \left( \frac{1}{2}\partial_{\mu}\phi \partial^{\mu}\phi - V(\phi)\right).$$
    Funciones de $n$ puntos
    $$ \langle O(\phi(x_1)) \cdots \rangle =  \frac{1}{Z}\int \mathcal{D}\phi \, O(\phi(x_1))\cdots e^{iS[\phi]}.$$
\end{frame}

\begin{frame}{Rotación de Wick. Espacio euclidiano}
    Rotación de Wick \\~
    $$ t \to i t.$$
    %Espaciotiempo de Minkowski $\to$ Espaciotiempo euclidiano \\~
    $$ \eta_{\mu\nu} \to \delta_{\mu\nu}$$
        
    Factor de peso 
        $$ \int \mathcal{D}\phi \, e^{iS} \to  \int \mathcal{D}\phi \, e^{-S_{\text{E}}},$$
        donde 
        $$ S_{\text{E}} = \int d^4 x \, \left( \frac{1}{2}\partial_{\mu}\phi \partial_{\mu}\phi + V(\phi)\right).$$
        Función de partición $ Z = \int \mathcal{D}s\, e^{-\beta H[s]}$.
        
        Teoría de campos $\to$ Física estadística. 
\end{frame}

\begin{frame}{Regularización de la retícula (Lattice Regularization).}
    $Z$ cantidad altamente divergente.\\~
    
    Necesita ser regularizada. \\~
    
    Regularización de la retícula:
    Discretizar el espaciotiempo
    \begin{itemize}
         \item $\displaystyle\partial_\mu\phi \to \frac{\phi(x + a\hat{\mu}) - \phi(x)}{a}$
         \item $\displaystyle\int d^4x \to a^4\sum_{x}$
    \end{itemize}
    
    Unidades de lattice $a = 1$. \\~
    
    Formalismo NO PERTURBATIVO.
\end{frame}

\begin{frame}
    $$\langle O \rangle = \sum_i^{N} O_i$$
    donde $O_i$ es generada con probabilidad $\frac{1}{Z}e^{-S_{\text{E}}}$
\end{frame}

\begin{frame}{Muestreo de importancia (importance sampling)}
    Para generar configuraciones, el algoritmo necesita cumplir
    \begin{itemize}
        \item Ergodicidad
        \item Balance detallado (detailed balance)
    \end{itemize}
\end{frame}

\begin{frame}{Algoritmos de actualización}
    Locales
    \begin{itemize}
        \item Metropolis
        \item Glauber
        \item Heatbath
        \item Hybrid Monte Carlo
        \item etc\dots
    \end{itemize}
    
    \ \\~
    
    Globales
    \begin{itemize}
        \item Wolff
        \item Swedesen-Wang
    \end{itemize}
\end{frame}

\begin{frame}{Algoritmo Metropolis}
    \begin{itemize}
        \item Escogemos un sitio de la retícula $x$.
        \item Generamos una propuesta del campo a actualizar $\phi_x \to \phi'_x$.
        \item Aceptamos la propuesta con probabilidad 
        $$ p = \min\left(1, \exp\left(-\Delta S\right)\right),$$
           donde
        $$ \Delta S = S[\phi'] - S[\phi].$$        
        (Generamos un numero aleatorio $r\in[0,1)$ si $r<p$, $\phi_x$ se actualiza a $\phi'_x$. Si $r>p$, rechazamos el cambio y conservamos a $\phi_x$.)
      
    \end{itemize}
\end{frame}

\begin{frame}{Análisis estadístico}

    \begin{itemize}
        \item Generación de configuraciones.
        \item Toma de mediciones.
        \item Cálculo de errores. Comúnmente \emph{Jackknife}.
        \item Interpretación de resultados.
    \end{itemize}

\end{frame}

\begin{frame}{Ejemplo: Modelo de Ising 2d}

Función hamiltoniana
$$ H[s] = -J\sum_{\langle ij \rangle} s_i s_j .$$
Función de partición
$$ Z = \sum_s e^{-\beta H[s]}.$$

\end{frame}

\begin{frame}{Ejemplo: Oscilador anarmónico cuántico 1d}

    Lagrangiano euclidiano

    $$ L_{\text{E}} = \frac{1}{2}\left(\partial_{\mu}x\right)^2 + \frac{1}{2}m^2 x^2 + \frac{1}{4}\lambda x^4,$$
con $\lambda > 0$. \\~

Del teorema del virial calculamos la energía del estado base $E_0$ 
$$ \langle E_0 \rangle = m^2\langle x \rangle + \lambda \langle x^3 \rangle. $$
\end{frame}

\begin{frame}{Ejemplo: Teoría de norma U(1) 2d}

Acción estándard de Wilson
$$ S[U] = \frac{\beta}{2}\sum_{x}\sum_{\mu < \nu} \text{Re} \left[1 - U_{\mu\nu} \right], $$
donde $\beta = 2/g^2$, con $g$ el acoplamiento de norma. 
\end{frame}

\begin{frame}{Ejemplo: Teoría de norma SU(2) 4d}
Acción estándard de Wilson
$$ S[U] = \frac{\beta}{4}\sum_{x}\sum_{\mu < \nu} \text{Re Tr} \left[\mathds{1} - U_{\mu\nu} \right] $$
\end{frame}


\begin{frame}{Ejemplo: Teoría de norma SU(3) 4d}
Acción estándard de Wilson
$$ S[U] = \frac{\beta}{6}\sum_{x}\sum_{\mu < \nu} \text{Re Tr} \left[\mathds{1} - U_{\mu\nu} \right] $$
\end{frame}


\begin{frame}{Resumen}

Regularización de \emph{lattice} es un enfoque no perturbativo.
    
\end{frame}

\begin{frame}{Referencias}

¡Siganme en mi GitHub \faGithub !
\url{https://github.com/JoseAntonioLattice}
\end{frame}


\end{document}
